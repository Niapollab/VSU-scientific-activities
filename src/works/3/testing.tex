% !TEX TS-program = lualatex
В рамках тестирования были созданы следующие модели с различными гиперпараметрами.

\begin{df}{|C|C|}{m}{Тестирование моделей с различными гиперпараметрами}{test-diff-params}\hline
    Модель & Число нейронов \\ \hline
    TD-Gammon Версия 1 & $40$ \\ \hline
    TD-Gammon Версия 2 & $80$ \\ \hline
\end{df}

Различие между моделями (\refdf{test-diff-params}) заключается в числе нейронов на скрытом слое. Проанализируем динамику развития нейронной сети в зависимости от числа нейронов, входящих в состав скрытого слоя.

\image{Сравнение двух моделей по победам за 10,000 игр}{model-comparing}{model-comparing}

На графике (\refimage{model-comparing}) видно, что модель второй версии является более совершенной с точки зрения количества выигранных матчей. Несмотря на увеличение размерности скрытого слоя в 2 раза, распределение выигрышей между моделями подчиняется закону нормального распределения 55 к 45 из 100.
