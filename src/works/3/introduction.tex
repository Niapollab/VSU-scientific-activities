% !TEX TS-program = lualatex
Существует множество разнообразных подходов к разработке искусственного интеллекта в играх: эвристические алгоритмы, алгоритмы, основанные на игровых деревьях, модели машинного обучения. Каждый конкретный тип служит для решения заранее определённой задачи. В таких играх, как <<Нарды>>, из-за обилия доступных вариантов развития партии очень распространены подходы, использующие модели машинного обучения. Строение и стратегии обучения сетей такого рода могут быть различными, однако наиболее популярными и эффективными подходами для обучения агентов в игре <<Нарды>> являются методы, основанные на обучении с подкреплением. Такой подход позволяет имитировать окружающую среду, которая выступает в роли учителя для нейронной сети. Одним из ключевых методов в этой области является обучение с временной разницей, сочетающее элементы динамического программирования и оценки состояния во времени.

В настоящей статье будут рассмотрены основные концепции обучения методом <<Временная разница>>, его преимущества и особенности при решении задач обучения с подкреплением на примере игры <<Нарды>>.
