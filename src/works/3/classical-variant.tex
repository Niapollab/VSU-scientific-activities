% !TEX TS-program = lualatex
Основная идея классического метода заключается в корректировке оценок состояния на основе разницы между текущей оценкой и последующей (\ref{td-error}), которая уже включает информацию о вознаграждении, полученном в будущем. Таким образом, математически метод $TD(0)$ или метод временных разниц может быть представлен следующим образом:

\pagebreak

\begin{equation}\label{td-error}
    \delta_t = R(s_t, a_t) + \gamma V(s_{t+1}) - V(s_t).
\end{equation}

\begin{equation}\label{td-base}
    V(s_t) \leftarrow V(s_t) + \alpha \delta_t.
\end{equation}

\begin{itemize}
    \item $\delta_t$ --- ошибка алгоритма на шаге времени $t$.
    \item $R(s_t, a_t)$ --- награда, полученная после выполнения действия $a_t$ в состоянии $s_t$.
    \item $\gamma$ --- коэффициент дисконтирования, который определяет, насколько важны будущие награды.
    \item $\alpha$ — скорость обучения, которая контролирует, насколько сильно обновляется текущее значение на основе ошибки алгоритма.
    \item $V(s_t)$ --- оценка состояния $s_t$ (до обновления).
    \item $V(s_{t+1})$ --- оценка следующего состояния $s_{t+1}$.
\end{itemize}
