% !TEX TS-program = lualatex
На вход должны быть предоставлены игровые позиции в формате, который может обрабатываться алгоритмом, позволяя модели принимать решения на основе изученных шаблонов и оценок состояния.

Всего на доске $N = 24$ различных пунктов и $P = 2$ число игроков. Вектор $\vec{c} = (c_1, c_2, c_3, c_4)$ кодирует каждый из $N$ доступных пунктов. $c_1, c_2, c_3$ принимают значения 0 или 1 в зависимости от числа шашек на пункте. $c_4 = \frac{n - 3}{2}, n \in [0, 15]$ (для уменьшения числа входных параметров). Например, $1000$ кодирует одну шашку на пункте, $1110$ — три шашки, $1113$ — восемь шашек. Так как шашки могут быть двух цветов, получаем: $|\vec{c}| \times N \times P = 4 \times 24 \times 2 = 192$ — ровно столько элементов необходимо для описания всех состояний шашек на $N$ пунктах.

Также закодируем число шашек на баре, как $\vec{b} = (b_1, b_2), b_i = \frac{n}{2}$. $\vec{d} = (d_1, d_2),  d_i = \frac{n}{15}$ --- число удаленных шашек, и текущий ход $\vec{t} = (t_1, t_2), t_i \in [0, 1]$. Таким образом, мощность итогового вектора входных параметров $|\vec{l_1}| = (|\vec{c}| \times N \times P) + |\vec{b}| + |\vec{d}| + |\vec{t}| = 192 + 2 + 2 + 2 = 198$.
