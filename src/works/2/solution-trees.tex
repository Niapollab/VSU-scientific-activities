% !TEX TS-program = lualatex
Подходы, основанные на деревьях решений, представляют набор алгоритмов и методологий, используемых для изучения и анализа обширного пространства решений, присущего игре. Эти методы используют древовидные структуры, называемые деревьями игры, для представления возможных последовательностей ходов и их результатов, что позволяет агентам искусственного интеллекта принимать обоснованные решения на основе стратегических соображений.

На \refimage{solution-tree-example} изображен пример узла некоторого дерева решений данной игры.

\dimage[width=0.5\textwidth]{Схематичное представление некоторого узла дерева решений для игры <<Нарды>>}{solution-tree-example}

Каждый узел в дереве представляет определенное игровое состояние, а ребра представляют возможные ходы, переводящие игру из одного состояния в другое.

В конечных игровых состояниях (узлах дерева) применяется оценочная функция для оценки желательности результата для игрока. Эта функция обычно учитывает такие факторы, как положение шашек, количество пунктов, контроль доски и другие стратегические соображения.

В рамках научного пособия \cite{ai-modern-approach} рассматривается широкий спектр тем, связанных с искусственным интеллектом в разнообразных играх, включая алгоритмы поиска, которые изложены ниже. В нем содержатся подробные объяснения, примеры и обсуждения этих методов в контексте абстрактных игровых агентов.
