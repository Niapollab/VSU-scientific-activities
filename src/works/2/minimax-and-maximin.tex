% !TEX TS-program = lualatex
Алгоритмы <<Минимакс>> и <<Максимин>> --- это алгоритмы принятия решений, используемые в теории игр и искусственном интеллекте. Они предназначены для определения оптимального хода игрока в игре с нулевой суммой, где выигрыш одного игрока означает проигрыш другого.

Оценка $u_i$ для $i$-го игрока может быть представлена в общем виде (\ref{maxmin}).
\begin{equation}\label{maxmin}
    u_i = \max_{a_i} \min_{a - i} u_i(a_i, a - i)
\end{equation}
$-i$ представляет индексы, отличные от $i$-го игрока, $a_i$ обозначает действие, предпринятое $i$-ым игроком, $a_i - 1$ действие, предпринятое другим игроком соответственно.

Алгоритм <<Максимин>> в формульной нотации похож на <<Минимакс>> за исключением, что стремится максимизировать минимально возможный выигрыш для игрока, предполагая, что противник будет действовать таким образом, чтобы свести его к минимуму, в то время как <<Минимакс>> стремится минимизировать максимальный потенциальный проигрыш для игрока, предполагая, что противник будет действовать таким образом, чтобы максимизировать его.
