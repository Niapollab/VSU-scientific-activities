% !TEX TS-program = lualatex
Алгоритм <<Поиск по дереву Монте-Карло>> является одним из самых популярных алгоритмов в области поиска решений на дереве. Отличительной особенностью данного алгоритма от других является обучаемость. Такое поведение достигается за счет введения специальных параметров в дерево решений, представляющих собой количество сыгранных и выигранных партий, соответственно. Схематично шаги данного алгоритма проиллюстрированы на \refimage{mcts-example}

\image[width=0.7\textwidth]{Схематичное представление алгоритма <<Поиск по дереву Монте-Карло>>}{mcts-example}{mcts-example}

Алгоритм состоит из 4 шагов:
\begin{enumerate}
    \item \textbf{Выбор}. Выбирается ход соперника в случае, если ход не существует --- он будет добавлен.
    \item \textbf{Расширение}. К ходу противника прибавляется узел, содержащих ход агента, а также <<нулевой результат>>.
    \item \textbf{Симуляция}. Вычисляется результат партии от текущего состояния до окончания игры.
    \item \textbf{Обратное распространение}. Результат симуляции распространяется от текущего узла до корневого.
\end{enumerate}
