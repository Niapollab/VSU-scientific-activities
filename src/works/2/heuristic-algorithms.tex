% !TEX TS-program = lualatex
В рамках темы настоящей работы будем считать, что эвристические алгоритмы --- это основанные на правилах стратегии или правила принятия решений, которые определяют поведение агента под управлением искусственного интеллекта, участвующего в игре. Их предназначение --- максимально приблизить ход к оптимальному на основе набора предопределенных правил, принципов или руководств. Оптимальным ходом будем считать такой ход, который приближает агента под управлением искусственного интеллекта к победе, на основе некоторой исходно заданной метрики.

Алгоритмы данного типа направлены на достижение баланса между скоростью и результативностью принятия решений, предоставляют практические и прагматичные стратегии для преодоления сложностей игрового процесса в игре <<Нарды>>. Многие подходы подробно описаны в литературе \cite{backgammon-strategies}. В рамках настоящей работы будут изложены лишь некоторые из них:

\begin{enumerate}
    \item \textbf{Удаление с доски}. Эвристика направлена на эффективное удаление шашек с доски на заключительном этапе игры. В ней приоритет отдается удалению шашек с пунктов дома игрока, где количество шашек максимально, чтобы сократить количество очков, необходимых для выигрыша. Такой подход позволяет не оставлять уязвимых мест и сохранять сбалансированную позицию на доске.
    \item \textbf{Позиционная эвристика}. Данная эвристика оценивает общее положение на доске для принятия решений. Она определяет приоритетность ходов, которые улучшают позицию игрока, одновременно препятствуя прогрессу противника. Учитываются такие факторы, как распределение шашек, контроль очков и потенциал для наращивания очков.
    \item \textbf{Эвристика контакта}. Эвристика направлена на усиление взаимодействия с шашками противника. Она определяет последовательность ходов, которые создают возможность поразить или заманить в ловушку шашки противника.
    \item \textbf{Гоночная эвристика}. Данная эвристика нацелена на быстрое продвижение шашки и их отбитие у соперника. Она определяет очередность ходов, которые ускоряют прогресс в гонке за отбивание шашек, учитывая такие факторы, как количество очков и позиция противника.
    \item \textbf{Эвристика <<Куб удвоения>>}. Эвристика позволяет оценить, следует ли предлагать или принимать удвоение во время игры, основываясь на таких факторах, как позиция игрока и количество занятых пунктов.
    \item \textbf{Безопасная эвристика}. Данная эвристика направлена на рассредоточение шашек и поддержание гибкой позиции, чтобы не оставлять пункты, занятые одной шашкой (блоты), уязвимыми для атаки.
\end{enumerate}
