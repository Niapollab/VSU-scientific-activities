% !TEX TS-program = lualatex
Нарды --- это игра для двух игроков, в которой каждый игрок стремится перемещать свои шашки по доске и в конечном итоге выбить их. В игре присутствуют элементы случайности из-за броска кубиков и стратегических решений о том, как передвигать шашки. Сложность возникает из-за множества возможных ходов и неопределенности, вносимой бросками игральных костей. Существует множество разновидностей игры <<Нарды>>. Наиболее известными считаются короткие и длинные нарды. В рамках настоящей работы, будет рассмотрена первая разновидность.

Целью данного исследования является изучение основных методологий, используемых при внедрении искусственного интеллекта в игру <<Нарды>>. В настоящей работе будут рассмотрены классические эвристические подходы, алгоритмы по обработке и обходу деревьев решений, а также методологии последних лет, основанные на нейронных сетях.
