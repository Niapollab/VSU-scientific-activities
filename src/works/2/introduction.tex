% !TEX TS-program = lualatex
Нарды, одна из старейших известных настольных игр, привлекающая значительное внимание исследователей искусственного интеллекта по всему миру. Существуют разнообразные подходы к разработке искусственного интеллекта для данной игры, так как интерес возник еще на заре исследований искусственного интеллекта для игр в целом.

Основополагающим является подход, основанный на правилах, разработанных экспертами, которые направляют процесс принятия решений в искомое состояние. Эти правила основаны на эвристике и стратегиях, полученных на основе человеческого опыта.

Широко используется подход обучения с подкреплением для обучения с помощью механизмов самостоятельной игры и обратной связи. Такой подход позволяет освоить оптимальные стратегии, максимизируя совокупное вознаграждение за несколько итераций.

Поиск по дереву методом Монте-Карло подход, сочетающий в себе поиск по дереву со случайным моделированием для оценки и выбора ходов.

Методы глубокого обучения, в частности сверточные нейронные сети и рекуррентные нейронные сети, используются для получения масштабных взаимосвязей и характеристик из состояний игры. Эти сети часто обучаются с использованием парадигм контролируемого обучения или обучения с подкреплением.

За прошедшие годы было разработано множество реализаций для освоения игры. В рамках настоящей работы будут рассмотрены существующие реализации программных решений, будут проведены тесты производительности и эффективности, будет приведена сравнительная таблица.
