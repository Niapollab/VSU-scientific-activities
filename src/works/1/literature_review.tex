% !TEX TS-program = lualatex
В работе <<Подход с использованием искусственного интеллекта к игре в нарды>> Чанг-Чжун Яна \cite{firstattempts} предлагается обзор научных достижений в области программирования игры <<Нарды>> на заре создания первых компьютерных программ. В тексте работы представлены структуры и алгоритмы, на основе которых может быть построена игра. Искусственный интеллект в статье рассматривается в виде полиномиальных функций, оценивающих ценность доски в случае некоторого хода.

В работе Исследование основных подходов к реализации искусственного интеллекта для игры <<Нарды>> \cite{myfirstwork}, мною были рассмотрены различные подходы к созданию искусственного интеллекта для игры <<Нарды>>. Рассмотрены эвристические алгоритмы, а также подходы теории игр и машинного обучения.

В статье Джеральда Тесауро <<TD-Gammon, самообучающаяся программа для игры в нарды, достигает уровня мастера игры>> \cite{annotationtdgammon} описывается компьютерная программа -- искусственный интеллект на базе нейронной сети под названием TD-Gammon. В основе лежит алгоритм обучения с подкреплением для извлечения уроков из результатов игры с самим собой. Несмотря на то, что данный алгоритм начинает со случайных начальных значений, он достигает удивительно высокого уровня игры в процессе обучения.

В статье Николаоса Папахристу и Иоанниса Рефанидиса <<Обучение нейронных сетей игре в нарды с использованием обучения с подкреплением>> \cite{annotationavli3d} рассматриваются две популярные в Греции вариации нард:
\begin{itemize}
    \item Плакото. В отличии от коротких нард, используется другая механика, в которой вместо того, чтобы выбивать шашку противника на бар, игрок можете <<прижать>> ее, встав на пункт и временно ограничивая ее движение.
    \item Февга. Длинные нарды с отличной стартовой позицией, запретом взятия двух фишек из головы при дубле, и невозможность заблокировать шашку соперника шестью идущими подряд шашками игрока.
\end{itemize}

Авторы обучают агентов, которые изучают функцию оценки игровой позиции для этих вариаций игр, и показывают, что результирующие агенты значительно превосходят Tavli3D -- одно из популярных приложений, реализующее вариации греческих нард с искусственным интеллектом.

В статье <<Нарды -- это сложно>> под авторством Р. Тил Уиттера \cite{backgammonishard} подробно анализируется вычислительная сложность нахождения решения для игры <<Нарды>>. Автор рассматривает различные виды игры, а также подходы к ее решению. В работе автор приходит к выводу, что в некоторых случаях сложность игры растет экспоненциально, так как из-за случайности игра может продолжаться бесконечно.
