% !TEX TS-program = lualatex
Игра в нарды предполагает наличие двух игроков. Для игры в нарды необходима специальная доска, состоящая из двадцати четырех узких треугольников, расположенных в форме подковы, называемых пунктами, тридцать шашек, по пятнадцать для каждого игрока и две игральные кости.

Пункты объединены в четыре группы по шесть пунктов в каждой. Эти группы называются -- дом, двор, дом противника, двор противника, соответственно. Пункты нумеруются для каждого игрока индивидуально, начиная с его дома. Последним является пункт под номер двадцать четыре, он также является первым пунктом для оппонента, соотвественно. Дом и двор разделены между собой планкой, которая выступает над игровым полем и называется баром.

Целью игры является перевод всех подконтрольных шашек в свой дом с последующим их снятием с доски. Первый игрок, который снял все свои шашки, выигрывает партию, состояние нечьи недопустимо.

Существует две основные разновидности игры короткие и длинные нарды.
