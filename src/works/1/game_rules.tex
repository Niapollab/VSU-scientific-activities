% !TEX TS-program = lualatex
Доска для игры в нарды состоит из двадцати четырех узких треугольников, расположенных в форме подковы, называемых пунктами. Каждый игрок управляет пятнадцатью шашками, различающимися по цвету. Первоначальная расстановка зависит от вида нард.

В рамках настоящей работы будут рассмотрены короткие нарды. Правые части доски при таком виде игры называются домами, а левые дворами. Игроки движутся по часовой стрелке. Порядковые номера пунктов начинаются с единицы и увеличиваются в сторону хода игрока. Цель состоит в том, чтобы вывести шашки за пределы доски, прежде чем это сделает соперник. В игре недостижимо состояние ничьи.

Каждый ход игрока начинается с броска двух шестигранных игральных костей. Исключение составляет первый ход, где каждый игрок бросает по одной кости, для того чтобы определить очередность ходов. Выпавшие числа определяют количество очков, которые могут пройти шашки игрока. Игрок может использовать каждую кость отдельно или объединить их для одного, более существенного хода. При объединении важно учитывать, что очки не суммируются. Таким образом, если оба хода по одной кости для шашки заблокированы -- ход по сумме недопустим. В случае выпадения дубля -- число ходов удваивается.

Ход допустим только в пункты, в которых шашек соперника меньше, чем две (одна или ни одной).  Когда допустимых позиций нет -- ход сгорает, в иных случаях, пропуск хода запрещен. Если игрок приземляется на пункт, занятый единственной шашкой противника, он отправляет ее на бар -- нейтральную зону за пределами основной доски. В следующий свой ход соперник должен вернуть ее в игру, прежде чем сможет сделать какие-либо другие ходы.

Когда все шашки находятся в доме, начинается процесс вывода шашек с доски. В случае, если по каким-то причинам одна из шашек покинула дом (отправилась на бар), она должна быть возвращена обратно. До этого момента процесс вывода недопустим.
