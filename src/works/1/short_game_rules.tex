% !TEX TS-program = lualatex
Перед игрой шашки устанавливаются в начальное положение. Также возможна расстановка, зеркально симметричная той, что приведена на \refimage{shortstartps} Дом в ней располагается слева, а двор -- справа, соответственно.

\dimage[width=0.5\textwidth]{Начальная позиция при игре в короткие нарды}{shortstartps}

Игроки поочередно бросают по две кости и выполняют ходы. Число на каждой кости показывает, на сколько пунктов игрок должен передвинуть свои шашки. Шашки движутся только в одном направлении: от пунктов с большими номерами к пунктам с меньшими (\refimage{shortturndr}). При этом применяются следующие правила:
\begin{itemize}
    \item Шашка может двигаться только на открытый пункт, то есть на такой, который не занят двумя или более шашками противоположного цвета.
    \item Числа на обеих костях составляют отдельные ходы.
\end{itemize}

\dimage[width=0.5\textwidth]{Направление движения шашек}{shortturndr}

Например, на костях выпало пять и три. Игроку доступны следующие комбинации ходов:
\begin{itemize}
    \item Пойти одной шашкой на три пункта, а другой -- на пять.
    \item Пойти одной шашкой сразу на восемь пунктов, при условии, что промежуточный пункт (на расстоянии три или пять шагов от начального пункта) открыт.
\end{itemize}

Игрок, у которого выпал дубль (одинаковое значение на обеих костях), играет каждое из чисел дважды. Например, если выпало две шестерки, то игрок должен сделать четыре хода на шесть пунктов, и он может передвинуть шашки в любой комбинации, как сочтет нужным. Игрок должен использовать оба выпавших числа (четыре, если выпал дубль), если они допускаются правилами. Когда можно сыграть только одно число, игрок обязан его сыграть. Если каждое из чисел по отдельности можно сыграть (но не оба вместе), игрок должен играть большее число. Случай, когда игрок не может сделать ход называется пропуском хода. В случае, если выпал дубль, если игрок не может использовать все четыре числа, он должен сыграть столько ходов, сколько возможно.

Блотом называется пункт, занятый только одной шашкой. Если шашка противоположного цвета останавливается на этом пункте, блот считается побитым и кладется на бар. Игрок обязан ввести в игру все шашки, находящиеся на баре, прежде чем совершит любой другой ход. Шашка вступает в игру, перемещаясь на пункт, соответствующий выброшенному значению кости. Если оба пункта, соответствующие значениям выброшенных костей, заняты, игрок пропускает ход. Если игрок может ввести только некоторые из своих шашек, он должен ввести все шашки, которые возможно, а затем пропустить оставшуюся часть хода.

После того, как все шашки будут введены с бара, неиспользованные значения костей можно использовать, как обычно.

Когда игрок привел все пятнадцать шашек в свой дом, он может начать процесс их снятия с доски. Игрок снимает шашку следующим образом: бросается пара костей, и шашки, которые стоят на пунктах, соответствующих выпавшим значениям, снимаются с доски. Например, если выпало число шесть, можно снять шашку с шестого пункта. Если на пункте, соответствующем выпавшей кости, нет ни одной шашки, игроку разрешается переместить шашку с пунктов, больших, чем выпавшее число. Если игрок может сделать какие-либо ходы, он не обязан снимать шашку с доски.

В стадии снятия шашек все шашки игрока должны находиться в его доме. Если шашка будет отправлена на бар в процессе снятия, то игрок должен привести шашку обратно в свой дом, прежде чем продолжит снятие. Тот, кто первый снял все шашки с доски, выигрывает партию.
