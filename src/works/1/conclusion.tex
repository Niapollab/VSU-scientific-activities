% !TEX TS-program = lualatex
В рамках настоящей работы была изложена постановка задачи, представлены аргументы в пользу актуальности и практической значимости, составлена математическая модель, а также сделан обзор литературы. Проведен анализ существующих подходов к решению задачи, таких как дерево решений и TD-Gammon, а также разработано алгоритмическое решение. Были определены критерии оценивая полученного решения и выполнено сравнение с существующими аналогами. Рассмотрена функциональность программ-аналогов, спроектирована архитектура собственной программной реализации разработанного решения. Было разработано программное решение с использованием языков программирования высокого уровня, проведено модульное, интеграционное и системное тестирование программного обеспечения. Неполадок в работе выявлено не было. Составлена схема публикации программного решения в открытый доступ, приложение было опубликовано.
