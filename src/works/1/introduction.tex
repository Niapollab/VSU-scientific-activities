% !TEX TS-program = lualatex
Искусственный интеллект в теории игр -- специально созданный агент, способный анализировать, обучаться и адаптироваться к стратегическим взаимодействиям с другими агентами с целью принятия оптимальных решений в игровой среде. Агентом является автономный объект (компьютерная программа или игрок), который наблюдает за средой и предпринимает действия для достижения определенных целей.

Искусственный интеллект играет важную роль в развитии технологий современного мира. Многие современные игры включают в себя передовые системы искусственного интеллекта, которые заставляют игроков мыслить критически и решать стратегические задачи. Этот опыт не только развлекает, но и улучшает когнитивные функции головного мозга, способствуя развитию интеллекта игрока и стимулируя навык принятия решений.

Так, игра <<Шахматы>>, которую часто называют <<королевской игрой>>, имеющая тысячелетнюю историю, была основана на идее моделирования военных действий. В первых версиях игры были представлены фигуры, представляющие различные виды индийских вооруженных сил, включая пехоту, кавалерию, слонов и колесницы. Для данной игры существует множество решений на базе искусственного интеллекта. Одни из самых популярных: AlphaZero (решение на базе нейронных сетей) и Stockfish (решение на базе анализа дерева игры в глубину).

В рамках настоящей работы будет рассмотрена игра <<Нарды>>. Корни игры можно проследить до древних цивилизаций Месопотамии, где археологические находки свидетельствуют о ее существовании с трехтысячных годов до нашей эры. В отличии от шахмат, современные правила нард содержат в себе некоторый элемент случайности. Это существенно сократило число исследований в области искусственного интеллекта.

Целью настоящей работы является создание программного продукта, предоставляющего искусственный интеллект на базе нейронных сетей для игры <<Нарды>>. В рамках поставленной цели, требуется решить следующие задачи:
\begin{enumerate}
    \item Изложить постановку задачи, составить математическую модель, исследовать актуальность и практическую значимость, сделать обзор литературы.
    \item Провести анализ существующих подходов к решению поставленной задачи, разработать алгоритмическое решение поставленной задачи.
    \item Определить критерии оценивая полученного алгоритмического решения, сравнить с существующими аналогами.
    \item Рассмотреть варианты использования программ аналогов, спроектировать архитектуру программной реализации разработанного решения.
    \item Разработать программное решение с использованием языков программирования высокого уровня, провести модульное, интеграционное и системное тестирование программного обеспечения.
    \item Составить схему публикации программного решения в открытый доступ, опубликовать приложение.
\end{enumerate}
