% !TEX TS-program = lualatex
Искусственный интеллект играет важную роль в развитии технологий современного мира. Многие современные игры включают в себя передовые системы искусственного интеллекта, которые заставляют игроков мыслить критически и решать стратегические задачи. Этот опыт не только развлекает, но и улучшает когнитивные функции головного мозга \cite{cognfunc}, способствуя развитию интеллекта игрока и стимулируя навык принятия решений. Игровые сценарии могут быть легко перенесены в реальные ситуации, предоставляя людям ценный опыт, применимый как в личном, так и в профессиональном контексте.

Игры часто служат средой моделирования, воспроизводящей события реального мира. Эта функция доказала свою полезность в различных профессиональных областях, таких как здравоохранение, авиация и обучение реагированию на чрезвычайные ситуации. Игровой искусственный интеллект используется для создания реалистичных симуляций, позволяя профессионалам практиковать и оттачивать свои навыки в безопасной виртуальной среде. Применение игрового искусственного интеллекта в сценариях обучения значительно повысило эффективность обучения, способствуя лучшей подготовке специалистов в различных отраслях.

Так, игра Chess (Шахматы), которую часто называют <<королевской игрой>>, имеющая тысячелетнюю историю, была основана на идее моделирования военных действий. В первых версиях игры были представлены фигуры, представляющие различные виды индийских вооруженных сил, включая пехоту, кавалерию, слонов и колесницы. Для данной игры существует множество решений на базе искусственного интеллекта. Одни из самых популярных: AlphaZero и Stockfish.

В рамках настоящей работы будет рассмотрена игра Backgammon (нарды). Корни игры можно проследить до древних цивилизаций Месопотамии, где археологические находки свидетельствуют о ее существовании с трехтысячных годов до нашей эры. В отличии от шахмат, современные правила нард содержат в себе некоторый элемент случайности. Это существенно сократило число исследований в области искусственного интеллекта.

Целью настоящей работы является исследование подходов к разработке искусственного интеллекта для игры Backgammon. В рамках поставленной цели, требуется решить следующие задачи:
\begin{enumerate}
\item Изложить постановку задачи.
\item Исследовать актуальность и практическую значимость.
\item Провести анализ подходов к решению поставленной задачи.
\end{enumerate}
