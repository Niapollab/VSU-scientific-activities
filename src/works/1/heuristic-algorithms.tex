% !TEX TS-program = lualatex
В настоящей работе будет рассмотрено несколько популярных подходов к разработке искусственного интеллекта для игры «Нарды». Простейшим подходом считается разработка искусственного интеллекта на основе эвристических алгоритмов --- последовательностей действий, основанных на некоторой эвристике. Такие алгоритмы называют приближенными, так как эвристика --- некоторое не относящееся к области исследования правило, позволяющее достичь заданных результатов при определённых обстоятельствах.

Такие алгоритмы не позволяют достигать оптимальных решений в большинстве игровых ситуаций, однако могут быть полезны в различных этюдах --- популярных состояниях игры, цитируемых в научной литературе. Например, эвристика <<Удаление с доски>> направлена на эффективное удаление шашек с доски на заключительном этапе игры. В ней приоритет отдается удалению шашек с пунктов дома игрока, где количество шашек максимально, чтобы сократить количество очков, необходимых для выигрыша.
