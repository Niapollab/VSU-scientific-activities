% !TEX TS-program = lualatex
\documentclass{vsureport}

% Copyright
\department{Кафедра вычислительной математики \\ и прикладных информационных технологий}
\title{Создание искусственного интеллекта на базе нейронных сетей для игры <<Нарды>>}
\titledesc{Отчет по НИР}
\direction{Направление 02.04.02 Фундаментальная информатика \\ и информационные технологии}
\profile{Профиль Машинное обучение \\ и интеллектуальные информационные технологии}
\headdep{Т. М. Леденева}
\headdepgrade{д.т.н., проф.}
\author{Н. В. Покатаев}
\sciadv{С. Н. Медведев}
\sciadvgrade{к.ф.-м.н., доц.}

\begin{document}
\maketitle

\tableofcontents

\nolabel{section}{Введение}
% !TEX TS-program = lualatex
Искусственный интеллект играет важную роль в развитии технологий современного мира. Многие современные игры включают в себя передовые системы искусственного интеллекта, которые заставляют игроков мыслить критически и решать стратегические задачи. Этот опыт не только развлекает, но и улучшает когнитивные функции головного мозга, способствуя развитию интеллекта игрока и стимулируя навык принятия решений. Игровые сценарии могут быть легко перенесены в реальные ситуации, предоставляя людям ценный опыт, применимый как в личном, так и в профессиональном контексте.

Игры часто служат средой моделирования, воспроизводящей события реального мира. Эта функция доказала свою полезность в различных профессиональных областях, таких как здравоохранение, авиация и обучение реагированию на чрезвычайные ситуации. Игровой искусственный интеллект используется для создания реалистичных симуляций, позволяя профессионалам практиковать и оттачивать свои навыки в безопасной виртуальной среде. Применение игрового искусственного интеллекта в сценариях обучения значительно повысило эффективность обучения, способствуя лучшей подготовке специалистов в различных отраслях.

Так, игра Chess (Шахматы), которую часто называют <<королевской игрой>>, имеющая тысячелетнюю историю, была основана на идее моделирования военных действий. В первых версиях игры были представлены фигуры, представляющие различные виды индийских вооруженных сил, включая пехоту, кавалерию, слонов и колесницы. Для данной игры существует множество решений на базе искусственного интеллекта. Одни из самых популярных: AlphaZero и Stockfish.

В рамках настоящей работы будет рассмотрена игра Backgammon (нарды). Корни игры можно проследить до древних цивилизаций Месопотамии, где археологические находки свидетельствуют о ее существовании с трехтысячных годов до нашей эры. В отличии от шахмат, современные правила нард содержат в себе некоторый элемент случайности. Это существенно сократило число исследований в области искусственного интеллекта.

Целью настоящей работы является исследование подходов к разработке искусственного интеллекта для игры Backgammon. В рамках поставленной цели, требуется решить следующие задачи:
\begin{enumerate}
\item Изложить постановку задачи.
\item Исследовать актуальность и практическую значимость.
\item Сделать обзор литературы.
\item Провести анализ подходов к решению поставленной задачи.
\end{enumerate}


\section{Формулировка задачи}
% !TEX TS-program = lualatex
В настоящей главе будут рассмотрены различные виды игры <<Нарды>>, их правила, поставлена задача о разработке искусственного интеллекта и составлен ее математический аппарат.


\subsection{Правила игры}
% !TEX TS-program = lualatex
Игра в нарды предполагает наличие двух игроков. Для игры в нарды необходима специальная доска, состоящая из двадцати четырех узких треугольников, расположенных в форме подковы, называемых пунктами, тридцать шашек, по пятнадцать для каждого игрока и две игральные кости.

Пункты объединены в четыре группы по шесть пунктов в каждой. Эти группы называются -- дом, двор, дом противника, двор противника, соответственно. Пункты нумеруются для каждого игрока индивидуально, начиная с его дома. Последним является пункт под номер двадцать четыре, он также является первым пунктом для оппонента, соотвественно. Дом и двор разделены между собой планкой, которая выступает над игровым полем и называется баром.

Целью игры является перевод всех подконтрольных шашек в свой дом с последующим их снятием с доски. Первый игрок, который снял все свои шашки, выигрывает партию, состояние нечьи недопустимо.

Существует две основные разновидности игры короткие и длинные нарды.


\subsubsection{Короткие нарды}
% !TEX TS-program = lualatex
Перед игрой шашки устанавливаются в начальное положение. Также возможна расстановка, зеркально симметричная той, что приведена на \refimage{shortstartps} Дом в ней располагается слева, а двор --- справа, соответственно.

\image[width=0.5\textwidth]{Начальная позиция при игре в короткие нарды}{short-backgammon}{shortstartps}

Игроки поочередно бросают по две кости и выполняют ходы. Число на каждой кости показывает, на сколько пунктов игрок должен передвинуть свои шашки. Шашки движутся только в одном направлении: от пунктов с большими номерами к пунктам с меньшими (\refimage{shortturndr}). При этом применяются следующие правила:
\begin{itemize}
    \item Шашка может двигаться только на открытый пункт, то есть на такой, который не занят двумя или более шашками противоположного цвета.
    \item Числа на обеих костях составляют отдельные ходы.
\end{itemize}

\image[width=0.5\textwidth]{Направление движения шашек}{short-backgammon-dir}{shortturndr}

Например, на костях выпало пять и три. Игроку доступны следующие комбинации ходов:
\begin{itemize}
    \item Пойти одной шашкой на три пункта, а другой --- на пять.
    \item Пойти одной шашкой сразу на восемь пунктов, при условии, что промежуточный пункт (на расстоянии три или пять шагов от начального пункта) открыт.
\end{itemize}

Игрок, у которого выпал дубль (одинаковое значение на обеих костях), играет каждое из чисел дважды. Например, если выпало две шестерки, то игрок должен сделать четыре хода на шесть пунктов, и он может передвинуть шашки в любой комбинации, как сочтет нужным. Игрок должен использовать оба выпавших числа (четыре, если выпал дубль), если они допускаются правилами. Когда можно сыграть только одно число, игрок обязан его сыграть. Если каждое из чисел по отдельности можно сыграть (но не оба вместе), игрок должен играть большее число. Случай, когда игрок не может сделать ход называется пропуском хода. В случае, если выпал дубль, если игрок не может использовать все четыре числа, он должен сыграть столько ходов, сколько возможно.

Блотом называется пункт, занятый только одной шашкой. Если шашка противоположного цвета останавливается на этом пункте, блот считается побитым и кладется на бар. Игрок обязан ввести в игру все шашки, находящиеся на баре, прежде чем совершит любой другой ход. Шашка вступает в игру, перемещаясь на пункт, соответствующий выброшенному значению кости. Если оба пункта, соответствующие значениям выброшенных костей, заняты, игрок пропускает ход. Если игрок может ввести только некоторые из своих шашек, он должен ввести все шашки, которые возможно, а затем пропустить оставшуюся часть хода.

После того, как все шашки будут введены с бара, неиспользованные значения костей можно использовать, как обычно.

Когда игрок привел все пятнадцать шашек в свой дом, он может начать процесс их снятия с доски. Игрок снимает шашку следующим образом: бросается пара костей, и шашки, которые стоят на пунктах, соответствующих выпавшим значениям, снимаются с доски. Например, если выпало число шесть, можно снять шашку с шестого пункта. Если на пункте, соответствующем выпавшей кости, нет ни одной шашки, игроку разрешается переместить шашку с пунктов, больших, чем выпавшее число. Если игрок может сделать какие-либо ходы, он не обязан снимать шашку с доски.

В стадии снятия шашек все шашки игрока должны находиться в его доме. Если шашка будет отправлена на бар в процессе снятия, то игрок должен привести шашку обратно в свой дом, прежде чем продолжит снятие. Тот, кто первый снял все шашки с доски, выигрывает партию.


\subsubsection{Длинные нарды}
% !TEX TS-program = lualatex
Пункт начального расположения шашек в данной версии игры, называется головой, а ход из начального положения называется <<из головы>> или <<взятие с головы>> (\refimage{longstartps}).

Правила во многом аналогичны коротким нардам за исключением нескольких пунктов:
\begin{itemize}
    \item За один ход с головы можно брать только одну шашку (исключение составляет первый бросок в партии, в котором можно снять две).
    \item Нельзя поставить свою шашку на пункт, занятый шашкой противника. Если шашки противника занимают шесть пунктов перед какой-нибудь шашкой, то ее называют запертой.
    \item Нельзя запереть все пятнадцать шашек противника (выстроить заграждение из шести подряд). Это допустимо только в том случае, если хотя бы одна шашка противника находится впереди этого заграждения.
\end{itemize}

\image[width=0.5\textwidth]{Начальная позиция при игре в длинные нарды}{long-backgammon-dir}{longstartps}


\subsection{Постановка задачи}
% !TEX TS-program = lualatex
В рамках настоящей работы требуется изложить постановку задачи, исследовать актуальность и практическую значимость, составить математическую модель, а также сделать обзор литературы. Необходимо провести анализ существующих подходов к решению задачи и разработать ее алгоритмическое решение. Определить критерии оценивая полученного решения и сравнить с существующими аналогами. Рассмотреть функциональность программ-аналогов, спроектировать архитектуру собственной программной реализации разработанного решения. Разработать программное решение с использованием языков программирования высокого уровня, провести модульное, интеграционное и системное тестирование программного обеспечения. Составить схему публикации программного решения в открытый доступ и опубликовать приложение.


\section{Актуальность и практическая значимость}
% !TEX TS-program = lualatex
Применение искусственного интеллекта в контексте игр привело к значительным достижениям в этой области.

Искусственный интеллект используются для анализа огромных объемов данных. Это способствует разработке передовых стратегий и тактик, ранее неприменимых на практике. Кит Вулси, вице-чемпион мира по нардам 1996 года, говорил, что только в некоторых технических моментах может претендовать на определённое преимущество перед таким искусственным интеллектом \cite{kitwoolsey}. Машина не подвержена эмоциональной предвзятости и учитывает только текущее состояние доски, а также потенциальные будущие результаты. Похожим образом обучаются автопилоты для самолетов и автомобилей, которые склонны минимизировать потери в случае аварии. Моральный аспект вопроса стал объектом для изучения в научных работах \cite{moralaspects}.

Искусственный интеллект может быть использован для тренировки специалистов. Программы могут анализировать игры, выявлять ошибки и предоставлять обратную связь по альтернативным ходам, способствуя обучению игроков на различных уровнях квалификации. Такие подходы также используются на производствах, для исправления бракованных продуктов в случае, если это возможно.

Исследования в области искусственного интеллекта для игр является актуальной задачей. Многие решения прошлого десятилетия могут быть улучшены благодаря вычислительным мощностям и открытиям современности. Например, некоторые представленные в настоящей работе решения, применимы только в контексте коротких нард с жестким набором правил, когда как другие имеют широкие границы применимости.


\section{Обзор научной литературы}
% !TEX TS-program = lualatex
В работе <<Подход с использованием искусственного интеллекта к игре в нарды>> Чанг-Чжун Яна \cite{firstattempts} предлагается обзор научных достижений в области программирования игры <<Нарды>> на заре создания первых компьютерных программ. В тексте работы представлены структуры и алгоритмы, на основе которых может быть построена игра. Искусственный интеллект в статье рассматривается в виде полиномиальных функций, оценивающих ценность доски в случае некоторого хода.

В работе Исследование основных подходов к реализации искусственного интеллекта для игры <<Нарды>> \cite{myfirstwork}, мною были рассмотрены различные подходы к созданию искусственного интеллекта для игры <<Нарды>>. Рассмотрены эвристические алгоритмы, а также подходы теории игр и машинного обучения.

В статье Джеральда Тесауро <<TD-Gammon, самообучающаяся программа для игры в нарды, достигает уровня мастера игры>> \cite{annotationtdgammon} описывается компьютерная программа --- искусственный интеллект на базе нейронной сети под названием TD-Gammon. В основе лежит алгоритм обучения с подкреплением для извлечения уроков из результатов игры с самим собой. Несмотря на то, что данный алгоритм начинает со случайных начальных значений, он достигает удивительно высокого уровня игры в процессе обучения.

В статье Николаоса Папахристу и Иоанниса Рефанидиса <<Обучение нейронных сетей игре в нарды с использованием обучения с подкреплением>> \cite{annotationavli3d} рассматриваются две популярные в Греции вариации нард:
\begin{itemize}
    \item Плакото. В отличии от коротких нард, используется другая механика, в которой вместо того, чтобы выбивать шашку противника на бар, игрок можете <<прижать>> ее, встав на пункт и временно ограничивая ее движение.
    \item Февга. Длинные нарды с отличной стартовой позицией, запретом взятия двух фишек из головы при дубле, и невозможность заблокировать шашку соперника шестью идущими подряд шашками игрока.
\end{itemize}

Авторы обучают агентов, которые изучают функцию оценки игровой позиции для этих вариаций игр, и показывают, что результирующие агенты значительно превосходят Tavli3D --- одно из популярных приложений, реализующее вариации греческих нард с искусственным интеллектом.

В статье <<Нарды --- это сложно>> под авторством Р. Тил Уиттера \cite{backgammonishard} подробно анализируется вычислительная сложность нахождения решения для игры <<Нарды>>. Автор рассматривает различные виды игры, а также подходы к ее решению. В работе автор приходит к выводу, что в некоторых случаях сложность игры растет экспоненциально, так как из-за случайности игра может продолжаться бесконечно.


\section{Анализ подходов к решению поставленной задачи}

\subsection{Эвристические алгоритмы}
% !TEX TS-program = lualatex
В рамках темы настоящей работы будем считать, что эвристические алгоритмы --- это основанные на правилах стратегии или правила принятия решений, которые определяют поведение агента под управлением искусственного интеллекта, участвующего в игре. Их предназначение --- максимально приблизить ход к оптимальному на основе набора предопределенных правил, принципов или руководств. Оптимальным ходом будем считать такой ход, который приближает агента под управлением искусственного интеллекта к победе, на основе некоторой исходно заданной метрики.

Алгоритмы данного типа направлены на достижение баланса между скоростью и результативностью принятия решений, предоставляют практические и прагматичные стратегии для преодоления сложностей игрового процесса в игре <<Нарды>>. Многие подходы подробно описаны в литературе \cite{backgammon-strategies}. В рамках настоящей работы будут изложены лишь некоторые из них:

\begin{enumerate}
    \item \textbf{Удаление с доски}. Эвристика направлена на эффективное удаление шашек с доски на заключительном этапе игры. В ней приоритет отдается удалению шашек с пунктов дома игрока, где количество шашек максимально, чтобы сократить количество очков, необходимых для выигрыша. Такой подход позволяет не оставлять уязвимых мест и сохранять сбалансированную позицию на доске.
    \item \textbf{Позиционная эвристика}. Данная эвристика оценивает общее положение на доске для принятия решений. Она определяет приоритетность ходов, которые улучшают позицию игрока, одновременно препятствуя прогрессу противника. Учитываются такие факторы, как распределение шашек, контроль очков и потенциал для наращивания очков.
    \item \textbf{Эвристика контакта}. Эвристика направлена на усиление взаимодействия с шашками противника. Она определяет последовательность ходов, которые создают возможность поразить или заманить в ловушку шашки противника.
    \item \textbf{Гоночная эвристика}. Данная эвристика нацелена на быстрое продвижение шашки и их отбитие у соперника. Она определяет очередность ходов, которые ускоряют прогресс в гонке за отбивание шашек, учитывая такие факторы, как количество очков и позиция противника.
    \item \textbf{Эвристика <<Куб удвоения>>}. Эвристика позволяет оценить, следует ли предлагать или принимать удвоение во время игры, основываясь на таких факторах, как позиция игрока и количество занятых пунктов.
    \item \textbf{Безопасная эвристика}. Данная эвристика направлена на рассредоточение шашек и поддержание гибкой позиции, чтобы не оставлять пункты, занятые одной шашкой (блоты), уязвимыми для атаки.
\end{enumerate}


\subsection{Дерево игры}
% !TEX TS-program = lualatex
Для комплексных игр, состоящих из большого числа вариантов хода, игровые сценарии принято представлять в удобном для анализа виде. Игровое дерево --- структура данных, предоставляющая графическое обозначение возможных ходов и исходов в последовательной, пошаговой или параллельной игре. Оно используется в теории игр и лежит в основе искусственного интеллекта во многих подходах.

Вершины дерева представляют различные точки принятия решений или состояния в игре. Каждая вершина представляет собой моментальный снимок игры в определенный момент. Корневая вершина представляет начальное состояние игры до того, как были сделаны какие-либо ходы.

Ребра соединяют узлы и представляют возможные ходы или переходы из одного состояния в другое. Каждое ребро соответствует решению, принятому игроком.

Листья являются конечными вершинами дерева и представляют терминальное состояния игры, такие как выигрыш, проигрыш или ничья. Они являются конечными результатами последовательности ходов. Пример игрового дерева для игры крестики-нолики представлен на \refimage{tttgame}

\image{Игровое дерево для игры крестики-нолики}{tic-tac-toe-game-tree}{tttgame}

Игровое дерево описывает состояние игры, но не определяет конечную стратегию, которую нужно предпринять для достижения желаемой игровой позиции. Для поиска решения на дереве используются специальные алгоритмы анализа. В настоящей работе будет приведено краткое описание некоторых из них.


\subsubsection{Алгоритм <<максимин>> и <<минимакс>>}
% !TEX TS-program = lualatex
Алгоритм <<Максимин>> (<<Минимакс>>) --- это алгоритм принятия решений, предназначенный для определения оптимального хода игрока в играх с нулевой суммой. Оценка определяет максимальную (минимальную) выгоду, которую игрок может получить, зная, что оппоненты разумны.

Максиминная (минимаксная) оценка $\underline{v_i}$ ($\overline{v_i}$) для $i$-го игрока может быть представлена в общем виде (\ref{maxmin}, \ref{minimax}).
\begin{equation}\label{maxmin}
    \underline{v_i} = \max_{a_i} \min_{a_{-i}} v_i(a_i, a_{-i})
\end{equation}
\begin{equation}\label{minimax}
    \overline{v_i} = \min_{a_{-i}} \max_{a_i} v_i(a_i, a_{-i})
\end{equation}
где ($-i$) представляет индекс, отличный от $i$-го игрока, $a_i$ обозначает действие, предпринятое $i$-ым игроком, $a_{-i}$ действие, предпринятое другим игроком соответственно.


\subsection{TD-Gammon}
% !TEX TS-program = lualatex
Одним из главных преимуществ полных игровых деревьев является возможность создать непобедимый искусственный интеллект. В начале десятилетия было успешно построено полное дерево для игры шашки \cite{checkersolved}.

Основной проблемой для разработки полных деревьев становится их экспоненциальный рост, поскольку каждый ход увеличивает число возможных исходов. В \refdf{gamedefparams} представлены мощности полных деревьев для некоторых популярных игр.

\begin{df}{|C|C|C|C|}{m}{Мощности полных деревьев для некоторых популярных игр}{gamedefparams}\hline
    Название игры & Степень ветвления дерева $b$ & Глубина дерева игры $d$ & Мощность дерева $b^d$ \\ \hline
    Крестики-нолики & $4$ & $9$ & $262144$  \\ \hline
    Шашки & $2.8$ & $70$ & $2 \times 10^{31}$  \\ \hline
    Шахматы & $35$ & $70$ & $1.22 \times 10^{128}$  \\ \hline
    Нарды & $250$ & $55$ & $7.70 \times 10^{131}$ \\ \hline
    Го & $250$ & $150$ & $4.91 \times 10^{359}$ \\ \hline
\end{df}

Это затрудняет полный анализ или представление всего дерева, особенно в сложных играх с большим количеством возможных ходов и исходов. Одним из возможных решений данной проблемы является аппроксимацию частей дерева, чтобы сделать вычисления выполнимыми. Однако это может привести к неоптимальным стратегиям или упущению важных игровых состояний.

TD-Gammon, сокращение от Temporal Difference Gammon, является знаковым достижением в области искусственного интеллекта и настольных игр в частности. Разработанный Джеральдом Тесауро в начале 1990-х годов, TD-Gammon продемонстрировал потенциал обучения нейронных сетей для освоения сложных стратегических игр.

В отличии от конкурентов в лице шахматных компьютеров того времени, искусственный интеллект TD-Gammon не анализировал дерево игры в глубину, а использовал обучение с подкреплением.

Идея обучения с подкреплением отличается от классической модели обучения с учителем. $i$-ым шагом обучения является выбор некоторого хода $a_i \in A$, где $A$ -- является множеством всех доступных ходов для некоторого состояния $s_i \in S$, где $S$ -- определяет множество всех доступных состояний.

В роли учителя выступает некоторая <<окружающая среда>>, оценивающая $a_i$-ый ход на основе $s_i$-ого состояния. Таким образом, находясь в состоянии $s_i$ искусственный интеллект выполняет поиск $a_i$-го хода, согласно некоторой оценке $r_i$ и получает от окружающей среды новую оценку $r_{i + 1}$, переходя в состояние $s_{i + 1}$ (\refimage{tdschema}).

\image{Схематичное представление процесса обучения с подкреплением}{td-schema}{tdschema}

В процессе реализации идеи возникают такие вопросы как: <<Что считать состоянием?>>, <<Как вычислить оценку?>>, <<Как распространить оценку от конечного состояния к начальному?>>. Существующие методы ответа на эти вопросы влияют на конечный результат, зависят от задачи и описаны в научных трудах \cite{tdlearn}.

Благодаря обширной самостоятельной игре и обучению на своих ошибках, искусственный интеллект достиг уровня игры, который превзошел уровень гроссмейстеров-людей. В 1992 году TD-Gammon выиграла чемпионат мира по нардам, продемонстрировав свое мастерство и стратегическое понимание игры.


\nolabel{section}{Заключение}
% !TEX TS-program = lualatex
В настоящей работе были рассмотрены классические эвристические подходы, алгоритмы по обработке и обходу деревьев решений, а также методологии последних лет, основанные на нейронных сетях. Каждая из этих подходов обладает уникальными преимуществами и возможностями, что делает данные методологии --- ценными инструментами в области искусственного интеллекта и принятия решений, в частности.


\bibliography{sources}

\end{document}
