% !TEX TS-program = lualatex
Для комплексных игр, состоящих из большого числа вариантов хода, игровые сценарии принято представлять в удобном для анализа виде. Игровое дерево --- структура данных, предоставляющая графическое обозначение возможных ходов и исходов в последовательной, пошаговой или параллельной игре. Оно используется в теории игр и лежит в основе искусственного интеллекта во многих подходах.

Вершины дерева представляют различные точки принятия решений или состояния в игре. Каждая вершина представляет собой моментальный снимок игры в определенный момент. Корневая вершина представляет начальное состояние игры до того, как были сделаны какие-либо ходы.

Ребра соединяют узлы и представляют возможные ходы или переходы из одного состояния в другое. Каждое ребро соответствует решению, принятому игроком.

Листья являются конечными вершинами дерева и представляют терминальное состояния игры, такие как выигрыш, проигрыш или ничья. Они являются конечными результатами последовательности ходов. Пример игрового дерева для игры крестики-нолики представлен на \refimage{tttgame}

\image{Игровое дерево для игры крестики-нолики}{tic-tac-toe-game-tree}{tttgame}

Игровое дерево описывает состояние игры, но не определяет конечную стратегию, которую нужно предпринять для достижения желаемой игровой позиции. Для поиска решения на дереве используются специальные алгоритмы анализа. В настоящей работе будет приведено краткое описание некоторых из них.
