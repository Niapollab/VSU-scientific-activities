% !TEX TS-program = lualatex
Игровое дерево -- это графическое представление возможных ходов и исходов в последовательной, пошаговой или параллельной игре. Оно используется в теории игр, лежит в основе искусственного интеллекта во многих подходах.

Вершины дерева представляют различные точки принятия решений или состояния в игре. Каждая вершина представляет собой моментальный снимок игры в определенный момент. Корневая вершина представляет начальное состояние игры до того, как были сделаны какие-либо ходы.

Ребра соединяют узлы и представляют возможные ходы или переходы из одного состояния в другое. Каждое ребро соответствует решению, принятому игроком.

Листья являются конечными вершинами дерева и представляют терминальное состояния игры, такие как выигрыш, проигрыш или ничья. Они являются конечными результатами последовательности ходов. Пример игрового дерева для игры крестики-нолики представлен на \refimage{tttgame}

\image{Игровое дерево для игры крестики-нолики}{tic-tac-toe-game-tree}{tttgame}

Игровые деревья особенно полезны при анализе и решении игр, особенно в контексте искусственного интеллекта. При анализе решения, для игр с нулевой суммой (один победитель -- один проигравший) часто используется алгоритм Minimax.

Идея алгоритма заключается в том, что оба игрока стремятся сделать наилучший ход, тем самым минимизируя шансы противника на победу. Алгоритм рекурсивно высчитывает всевозможные ходы от имени обоих игроков до тех пор, пока не достигнет определенной глубины. Затем метод оценивает текущую позицию, вычисляя ее качество или вес. Оценка $u_i$ для $i$-го игрока может быть представлена в общем виде (\ref{maxmin}).
\begin{equation}\label{maxmin}
    u_i = \max_{a_i} \min_{a - i} u_i(a_i, a - i)
\end{equation}
$-i$ представляет индексы, отличные от $i$-го игрока, $a_i$ обозначает действие, предпринятое $i$-ым игроком, $a_i - 1$ действия, предпринятые другими игроками соответственно.
