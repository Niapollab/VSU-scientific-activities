% !TEX TS-program = lualatex
Пункт начального расположения шашек в данной версии игры, называется головой, а ход из начального положения называется <<из головы>> или <<взятие с головы>> (\refimage{longstartps}).

Правила во многом аналогичны коротким нардам за исключением нескольких пунктов:

\begin{itemize}
    \item За один ход с головы можно брать только одну шашку (исключение составляет первый бросок в партии, в котором можно снять две).
    \item Нельзя поставить свою шашку на пункт, занятый шашкой противника. Если шашки противника занимают шесть пунктов перед какой-нибудь шашкой, то ее называют запертой.
    \item Нельзя запереть все пятнадцать шашек противника (выстроить заграждение из шести подряд). Это допустимо только в том случае, если хотя бы одна шашка противника находится впереди этого заграждения.
\end{itemize}

\image[width=0.5\textwidth]{Начальная позиция при игре в длинные нарды}{long-backgammon-dir}{longstartps}
